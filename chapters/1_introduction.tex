\section[Dependable Cloud Computing with OpenStack \texorpdfstring{{\textbf{\tiny \enspace (JE, SK, NK)}}}{}]{Dependable Cloud Computing with OpenStack}
\sectionmark{Dependable Cloud Computing with OpenStack}
\label{introduction}

As cloud computing becomes more and more popular, there are an increasing number of  implementations to offer various cloud-service models like infrastructure as a service (IaaS), platform as a service (PaaS) or software as a service (SaaS). While many companies offer commercial solutions like the Amazon Elastic Compute Cloud (EC2) or HP Helion, there are also open source alternatives that can be freely installed and configured to meet the needs of ones projects with respect to the underlying hardware available.\\

One of the open source variants for achieving a cloud computing system is OpenStack. OpenStack is a cloud software stack which allows for offering infrastructure as a service, almost independent of the underlying hardware setup. OpenStack itself can be seen as a collection of services that can be setup depending on the specifications of the planned use cases. The most important components that OpenStack offers are the networking, virtualization and storage services. Furthermore, it is possible to add further components to an OpenStack installation, e.g., services that handle billing or allow for object storage in the cloud. \\

This report is will describe the results of the masters project ``Dependable Cloud Computing with OpenStack'' of the summer term 2015 at the Hasso Plattner Institute Potsdam. An important factor, especially in cloud computing, is dependability. When offering such a service, it should be highly available, meaning that the system should be continuously operational without failing. Therefore, our main task was to analyze dependability mechanisms of OpenStack. To do this, we chose to manually setup a clean OpenStack environment (i.e. none provided by a third party like HP Helion) on which we would be able to run the specific analyses. We turned this manual installation into an automated one in order to simplify and speed up the process of setting up a working OpenStack test environment and making the resulting analyses of dependability reproducible. Since no OpenStack installation is exactly the same, the reproducibility of the results of such analyses is not an easy feat. We tackle this issue by making the test environment for the experiments completely virtual. Thus we circumvent tedious hardware setup, hardware errors that disturb the experiments. This also allows an fast rerun of the experiments and switching off network infrastructure. \\

In Chapter~\ref{related} of this report we will list other possibilities of deploying an OpenStack system and give insights into why we chose to create our own automated installation. Further, we will describe related work in the field of dependability analyses on OpenStack. Chapter~\ref{dependability} will contain theoretical background of the term dependability applied to the OpenStack domain. We will define what we understand under dependability and how this affects the analyses we run on the test environment. The test environment will be described in detail in Chapter~\ref{environment}. We will explain the reasons for our choice of test environment and discuss dependability testing. Chapter~\ref{installing} will contain an overview of the technologies used for the automated installation of OpenStack. In Chapter~\ref{experiments}, we will describe a set of automated dependability analyses on OpenStack, for which we will use the term ``experiments''. Our conclusions as well as potential future work will be given in Chapter~\ref{conclusion}. Furthermore, we will include a detailed documentation of our system in Appendix~\ref{appendix}. \\